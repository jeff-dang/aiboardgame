\documentclass[12pt, titlepage]{article}

\usepackage{svg}
\usepackage{fullpage}
\usepackage[round]{natbib}
\usepackage{multirow}
\usepackage{booktabs}
\usepackage{tabularx}
\usepackage{graphicx}
\usepackage{float}
\usepackage{hyperref}
\hypersetup{
    colorlinks,
    citecolor=blue,
    filecolor=black,
    linkcolor=red,
    urlcolor=blue
}

\input{Comments}
%% Common Parts

\newcommand{\progname}{ProgName} % PUT YOUR PROGRAM NAME HERE %Every program
                                % should have a name

\usepackage{hyperref}
    \hypersetup{colorlinks=true, linkcolor=blue, citecolor=blue, filecolor=blue,
                urlcolor=blue, unicode=false}
    \urlstyle{same}
                                

\newcounter{acnum}
\newcommand{\actheacnum}{AC\theacnum}
\newcommand{\acref}[1]{AC\ref{#1}}

\newcounter{ucnum}
\newcommand{\uctheucnum}{UC\theucnum}
\newcommand{\uref}[1]{UC\ref{#1}}

\newcounter{mnum}
\newcommand{\mthemnum}{M\themnum}
\newcommand{\mref}[1]{M\ref{#1}}

\begin{document}

\title{Module Guide for SE 4G06 \\ An AI-based Approach to Designing Board Games} 
\author{\authname{}}
\date{\today}
\author{\authname}


\maketitle

\pagenumbering{roman}

\section{Revision History}

\begin{tabularx}{\textwidth}{p{3cm}p{2cm}X}
\toprule {\bf Date} & {\bf Version} & {\bf Notes}\\
\midrule
April 3rd & 1.0 & Split design doc into 3 documents from Rev0 feedback for Rev1\\
April 5th & 2.0 & Add Module Decomposition Game Engine and Data Visualizer by [Tianzheng Mai]\\
\bottomrule
\end{tabularx}

\newpage

\section{Reference Material}

This section records information for easy reference.

\subsection{Abbreviations and Acronyms}

\renewcommand{\arraystretch}{1.2}
\begin{tabular}{l l} 
  \toprule		
  \textbf{symbol} & \textbf{description}\\
  \midrule 
  AC & Anticipated Change\\
  DAG & Directed Acyclic Graph \\
  M & Module \\
  MG & Module Guide \\
  OS & Operating System \\
  R & Requirement\\
  SC & Scientific Computing \\
  SRS & Software Requirements Specification\\
  Board Gamers & Development Team\\
  UC & Unlikely Change \\
  AI & Artificial Intelligence \\
  A & Assumption \\
  LC & Likely Change\\
  FR & Functional Requirement \\
  NFR & Non Functional Requirement \\
  FSM & Finite State Machine \\ 
  TA & Teaching Assistant \\
  \bottomrule
\end{tabular}\\



\newpage

\tableofcontents

\listoftables

\listoffigures

\newpage

\pagenumbering{arabic}

\section{Introduction}

Decomposing a system into modules is a commonly accepted approach to developing
software.  A module is a work assignment for a programmer or programming
team~\citep{ParnasEtAl1984}.  We advocate a decomposition
based on the principle of information hiding~\citep{Parnas1972a}.  This
principle supports design for change, because the ``secrets'' that each module
hides represent likely future changes.  Design for change is valuable in SC,
where modifications are frequent, especially during initial development as the
solution space is explored.  

Our design follows the rules layed out by \citet{ParnasEtAl1984}, as follows:
\begin{itemize}
\item System details that are likely to change independently should be the
  secrets of separate modules.
\item Each data structure is implemented in only one module.
\item Any other program that requires information stored in a module's data
  structures must obtain it by calling access programs belonging to that module.
\end{itemize}

After completing the first stage of the design, the Software Requirements
Specification (SRS), the Module Guide (MG) is developed~\citep{ParnasEtAl1984}. The MG
specifies the modular structure of the system and is intended to allow both
designers and maintainers to easily identify the parts of the software.  The
potential readers of this document are as follows:

\begin{itemize}
\item New project members: This document can be a guide for a new project member
  to easily understand the overall structure and quickly find the
  relevant modules they are searching for.
\item Maintainers: The hierarchical structure of the module guide improves the
  maintainers' understanding when they need to make changes to the system. It is
  important for a maintainer to update the relevant sections of the document
  after changes have been made.
\item Designers: Once the module guide has been written, it can be used to
  check for consistency, feasibility, and flexibility. Designers can verify the
  system in various ways, such as consistency among modules, feasibility of the
  decomposition, and flexibility of the design.
\end{itemize}

The rest of the document is organized as follows. Section
\ref{SecChange} lists the anticipated and unlikely changes of the software
requirements. Section \ref{SecMH} summarizes the module decomposition that
was constructed according to the likely changes. Section \ref{SecConnection}
specifies the connections between the software requirements and the
modules. Section \ref{SecMD} gives a detailed description of the
modules. Section \ref{SecTM} includes two traceability matrices. One checks
the completeness of the design against the requirements provided in the SRS. The
other shows the relation between anticipated changes and the modules. Section
\ref{SecUse} describes the use relation between modules.

\section{Anticipated and Unlikely Changes} \label{SecChange}

This section lists possible changes to the system. According to the likeliness
of the change, the possible changes are classified into two
categories. Anticipated changes are listed in Section \ref{SecAchange}, and
unlikely changes are listed in Section \ref{SecUchange}.

\subsection{Anticipated Changes} \label{SecAchange}

Anticipated changes are the source of the information that is to be hidden
inside the modules. Ideally, changing one of the anticipated changes will only
require changing the one module that hides the associated decision. The approach
adapted here is called design for
change.

\begin{description}
\item[\refstepcounter{acnum} \actheacnum \label{acAI}:] The specific AI agent learning method that will be used to train the AI Agents.
\item[\refstepcounter{acnum} \actheacnum \label{acAINum}:] The number of AI agents playing a given game engine.
\item[\refstepcounter{acnum} \actheacnum \label{acGE}:] The specific game engine that will be integrated and developed alongside the framework 
\item[\refstepcounter{acnum} \actheacnum \label{acDV}:] The graphs and charts to be generated by the data visualizer.
\end{description}

\subsection{Unlikely Changes} \label{SecUchange}

The module design should be as general as possible. However, a general system is
more complex. Sometimes this complexity is not necessary. Fixing some design
decisions at the system architecture stage can simplify the software design. If
these decision should later need to be changed, then many parts of the design
will potentially need to be modified. Hence, it is not intended that these
decisions will be changed.

\begin{description}
\item[\refstepcounter{ucnum} \uctheucnum \label{ucAILib}:] The entire AI library used to implement the AI Agents.
\item[\refstepcounter{ucnum} \uctheucnum \label{ucGE}:] The algorithm used to accept and execute actions on the game engine.
\item[\refstepcounter{ucnum} \uctheucnum \label{ucDV}:] The format of the output log from the Game Engine to the Data Visualizer.
\end{description}
\section{Module Hierarchy} \label{SecMH}

This section provides an overview of the module design. Modules are summarized
in a hierarchy decomposed by secrets in Table \ref{TblMH}. The modules listed
below, which are leaves in the hierarchy tree, are the modules that will
actually be implemented.

\begin{description}
\item [\refstepcounter{mnum} \mthemnum \label{mAI}:] AI Agent Module
\item [\refstepcounter{mnum} \mthemnum \label{mGE}:] Game Environment Module
\item [\refstepcounter{mnum} \mthemnum \label{mAct}:] Actions (Commands) Module
\item [\refstepcounter{mnum} \mthemnum \label{mGL}:] Game Loop Module
\item [\refstepcounter{mnum} \mthemnum \label{mJSON}:] JSON Module
\item [\refstepcounter{mnum} \mthemnum \label{mGr}:] Graph Module
\item [\refstepcounter{mnum} \mthemnum \label{mJSONDP}:] JSON Data Parser Module
\end{description}
\subsection{Module Hierarchy}
\begin{table}[h!]
\centering
\begin{tabular}{p{0.1\textwidth} p{0.2\textwidth} p{0.6\textwidth}}
\toprule
\textbf{Module Type} & \textbf{Module Name} & \textbf{Module Description}\\
\midrule
AI & \multirow{1}{0.3\textwidth}{AI Agent Module} & Trains AI Agents on the game and generates a policy\\
AI & \multirow{1}{0.3\textwidth}{Game Environment Module} & Receives input from AI Agents to take action on the game \\
\midrule
GE & \multirow{1}{0.24\textwidth}{Actions (Commands) Module} & {\textcolor{white}{\_\_\_}}Describes the possible game moves that the AI Agents are able to take \\
GE & \multirow{1}{0.3\textwidth}{Game Loop \\Module}  & Continues the game loop for the game and checks if the game-over condition has been fulfilled or not. \\
\midrule

DV & \multirow{1}{0.3\textwidth}{JSON Module} & Responsible for recording each AI Agents moves and observation space and putting them into a JSON file. \\
DV & \multirow{1}{0.3\textwidth}{Graph Module} & Produce a graph selected by the user.  \\
DV & \multirow{1}{0.3\textwidth}{JSON Data Parser \\ Module} & Parses JSON files with AI Agents' move history \\\\
\bottomrule

\end{tabular}
\caption{Module Hierarchy}
\label{TblMH}
\end{table}

\section{Connection Between Requirements and Design} \label{SecConnection}

The design of the system is intended to satisfy the requirements developed in
the SRS. In this stage, the system is decomposed into modules. The connection
between requirements and modules is listed in Table~\ref{TblRT}.

\section{Module Decomposition} \label{SecMD}

Modules are decomposed according to the principle of ``information hiding''
proposed by \citet{ParnasEtAl1984}. The \emph{Secrets} field in a module
decomposition is a brief statement of the design decision hidden by the
module. The \emph{Services} field specifies \emph{what} the module will do
without documenting \emph{how} to do it. For each module, a suggestion for the
implementing software is given under the \emph{Implemented By} title. If the
entry is \emph{OS}, this means that the module is provided by the operating
system or by standard programming language libraries.  \emph{\progname{}} means the
module will be implemented by the \progname{} software.  Implemented by AI libraries means the module will be provided by the AI Library of choice in the system.

Only the leaf modules in the hierarchy have to be implemented. If a dash
(\emph{--}) is shown, this means that the module is not a leaf and will not have
to be implemented.

\subsection{AI Modules}
\begin{description}
\item[Secrets:] The algorithms and data structures that facilitate the learning and training of the AI.
\item[Services:] Exposes methods for our code to learn and train, so the system does not need to implement from scratch.
\item[Implemented By:] \progname{} 
\end{description}

\subsubsection{AI Agent Module (\mref{mAI})}
\begin{description}
\item[Secrets:] The data structures and algorithms used to train the AI's
\item[Services:] Exposes methods for training.
\item[Implemented By:] AI Library
\end{description}
\subsubsection{Game Environment Module (\mref{mGE})}
\begin{description}
\item[Secrets:] None
\item[Services:] Interfaces with the AI Libraries to connect the Game Engine and the AI Agents
\item[Implemented By:] \progname{}
\end{description}

\subsection{Game Engine Module}
\begin{description}
\item[Secrets:] The model provides information on the current state of the game and possible actions to the AI Agents, and processes their response to drive the game. 
\item[Services:] Continues the game loop for the game and check if the game-over condition has been fulfilled. 
\item[Implemented By:] \progname{}
\end{description}
\subsubsection{Action (Command) Module (\mref{mAct})}
\begin{description}
\item[Secrets:] The model refers to the moves that an AI Agent can take. 
\item[Services:] List all valid game moves of the AI Agent. 
\item[Implemented By:] \progname{}
\end{description}
\subsubsection{Game Loop Module (\mref{mGL})}
\begin{description}
\item[Secrets:] The engine manages the overall game logic and states. 
\item[Services:] keeps track of the current state of the game and respond to AI Agent's input in a continuous loop while the game is running. 
\item[Implemented By:] \progname{}
\end{description}

\subsection{Data Visualizer Module}

\begin{description}
\item[Secrets:] The user interface is used to convey complicated game data in various graphical models. 
\item[Services:] Provides visual representations of data to help users understand complex information from the game output. 
\item[Implemented By:] \progname{}
\end{description}
\subsubsection{JSON Module (\mref{mJSON})}
\begin{description}
\item[Secrets:] The method used to record the JSON data of each AI Agent's moves and observation space. 
\item[Services:] Gain the information from the simulations and store them in a JSON file for further use in other modules. 
\item[Implemented By:] \progname{}
\end{description}
\subsubsection{Graph Module (\mref{mGr})}
\begin{description}
\item[Secrets:]  The model is used to define the statistical information, display the distribution, trends, and common winning paths of data points and make comparisons in various graphical models.
\item[Services:] Utilize the output data from the game environment and display them in diverse charts.
\item[Implemented By:] \progname{}
\end{description}
\subsubsection{JSON Data Parser Module (\mref{mJSONDP})}
\begin{description}
\item[Secrets:] The bridge used to connect the game Engine and data visualization model. 
\item[Services:] Fetch data from the JSON log file in the game for data visualization. 
\item[Implemented By:] \progname{}
\end{description}

\section{Traceability Matrix} \label{SecTM}

This section shows two traceability matrices: between the modules and the
requirements and between the modules and the anticipated changes.

% the table should use mref, the requirements should be named, use something
% like fref
\section{Module Traceability}
\begin{table}[h!]
\centering
\begin{tabular}{p{0.3\textwidth} p{0.6\textwidth}}
\toprule
\textbf{Module Name} & \textbf{Requirements}\\
\midrule

\multirow{1}{0.4\textwidth}{AI Agent Module} & FR1 FR2 FR4 FR6 NFR1 NFR3 NFR4 NFR5\\
\multirow{1}{0.4\textwidth}{Game Environment Module} & FR3 FR5 NFR1 NFR3 NFR5\\
\midrule
\multirow{1}{0.4\textwidth}{Action (Command) Module} & FR4 FR7 FR11 NFR3 NFR6\\
\multirow{1}{0.4\textwidth}{Game Loop Module} & FR7 FR8 FR9 FR10 FR11 NFR1 NFR3 NFR4 NFR5 NFR6\\
\midrule

\multirow{1}{0.4\textwidth}{JSON Module} & FR3 FR13 NFR3\\
\multirow{1}{0.4\textwidth}{Graph Module} & FR12 NFR2 NFR7\\
\multirow{1}{0.4\textwidth}{JSONDataParser Module} & FR13 FR14 FR15\\
\bottomrule

\end{tabular}
\caption{Module Traceability}
\label{TblRT}
\end{table}

\begin{table}[H]
\centering
\begin{tabular}{p{0.2\textwidth} p{0.6\textwidth}}
\toprule
\textbf{AC} & \textbf{Modules}\\
\midrule
\acref{acAI} & \mref{mAI}\\
\acref{acAINum} & \mref{mAI} \mref{mGE}\\
\acref{acGE} & \mref{mAct} \mref{mGL}\\
\acref{acDV} & \mref{mGr}\\
\bottomrule
\end{tabular}
\caption{Trace Between Anticipated Changes and Modules}
\label{TblACT}
\end{table}

\section{Use Hierarchy Between Modules} \label{SecUse}

In this section, the uses hierarchy between modules is
provided. \citet{Parnas1978} said of two programs A and B that A {\em uses} B if
correct execution of B may be necessary for A to complete the task described in
its specification. That is, A {\em uses} B if there exist situations in which
the correct functioning of A depends upon the availability of a correct
implementation of B.  Figure \ref{FigUH} illustrates the use relation between
the modules. It can be seen that the graph is a directed acyclic graph
(DAG). Each level of the hierarchy offers a testable and usable subset of the
system, and modules in the higher level of the hierarchy are essentially simpler
because they use modules from the lower levels.

\begin{figure}[H]
\centering
\includesvg[width=0.6\textwidth]{usehierarchy.drawio.svg}
\caption{Use hierarchy among modules}
\label{FigUH}
\end{figure}

%\section*{References}
\newpage{}

\bibliographystyle {plainnat}
\bibliography{References}


\end{document}