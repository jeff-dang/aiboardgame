\documentclass[12pt, titlepage]{article}

\usepackage{amsmath, mathtools}

\usepackage[round]{natbib}
\usepackage{amsfonts}
\usepackage{amssymb}
\usepackage{graphicx}
\usepackage{colortbl}
\usepackage{xr}
\usepackage{hyperref}
\usepackage{longtable}
\usepackage{xfrac}
\usepackage{tabularx}
\usepackage{float}
\usepackage{siunitx}
\usepackage{booktabs}
\usepackage{multirow}
\usepackage[section]{placeins}
\usepackage{caption}
\usepackage{fullpage}

\hypersetup{
bookmarks=true,     % show bookmarks bar?
colorlinks=true,       % false: boxed links; true: colored links
linkcolor=red,          % color of internal links (change box color with linkbordercolor)
citecolor=blue,      % color of links to bibliography
filecolor=magenta,  % color of file links
urlcolor=cyan          % color of external links
}

\usepackage{array}

\externaldocument{SRS}

%% Comments

\usepackage{color}

\newif\ifcomments\commentstrue %displays comments
%\newif\ifcomments\commentsfalse %so that comments do not display

\ifcomments
\newcommand{\authornote}[3]{\textcolor{#1}{[#3 ---#2]}}
\newcommand{\todo}[1]{\textcolor{red}{[TODO: #1]}}
\else
\newcommand{\authornote}[3]{}
\newcommand{\todo}[1]{}
\fi

\newcommand{\wss}[1]{\authornote{blue}{SS}{#1}} 
\newcommand{\plt}[1]{\authornote{magenta}{TPLT}{#1}} %For explanation of the template
\newcommand{\an}[1]{\authornote{cyan}{Author}{#1}}

%% Common Parts

\newcommand{\progname}{SE 4G06} % PUT YOUR PROGRAM NAME HERE
\newcommand{\authname}{Team \#6, Board Gamers
\\ Ilao Michael, ilaom
\\ Bedi Hargun, bedih
\\ Dang Jeffrey, dangj12
\\ Ada Jonah, karaatan
\\ Mai Tianzheng, mait6} % AUTHOR NAMES                  

\usepackage{hyperref}
    \hypersetup{colorlinks=true, linkcolor=blue, citecolor=blue, filecolor=blue,
                urlcolor=blue, unicode=false}
    \urlstyle{same}
                                

\begin{document}

\title{Module Interface Specification for SE 4G06 \\ An AI-based Approach to Designing Board Games} 
\author{\authname{}}
\date{\today}
\author{\authname}

\date{\today}

\maketitle

\pagenumbering{roman}

\section{Revision History}

\begin{tabularx}{\textwidth}{p{3cm}p{2cm}X}
\toprule {\bf Date} & {\bf Version} & {\bf Notes}\\
\midrule
April 3rd & 1.0 & Split up Rev 0 into 3 documents and implemented feedback\\
\bottomrule
\end{tabularx}

~\newpage

\section{Symbols, Abbreviations and Acronyms}
\subsection{Abbreviations and Acronyms}
\renewcommand{\arraystretch}{1.2}
\begin{tabular}{l l} 
  \toprule		
  \textbf{symbol} & \textbf{description}\\
  \midrule 
  SRS & Software Requirements Specification\\
  AI & Artificial Intelligence \\
  A & Assumption \\
  LC & Likely Change\\
  FR & Functional Requirement \\
  NFR & Non Functional Requirement \\
  FSM & Finite State Machine \\ 
  TA & Teaching Assistant \\
  \bottomrule
\end{tabular}\\
\newpage

\tableofcontents
\listoftables

\newpage

\pagenumbering{arabic}

\section{Introduction}

The following document details the Module Interface Specifications for An AI-based Approach to Designing Board Games, a system that simulates thousands of board game simulations using AI to visualize winning strategies for game designers to help balance their game. 

Complementary documents include the System Requirement Specifications and Module Guide.  The full documentation and implementation can be found \href{https://github.com/Dorps/aiboardgame/tree/main/docs/Design}{here}, under \href{https://github.com/Dorps/aiboardgame/tree/main/docs/Design/MG.pdf}{MG} and \href{https://github.com/Dorps/aiboardgame/tree/main/docs/DES.pdf}{DES}.

See SRS Documentation \href{https://github.com/Dorps/aiboardgame/blob/main/docs/SRS}{here}, for full list of requirements.

\section{Notation}

The structure of the MIS for modules comes from \citet{HoffmanAndStrooper1995},
with the addition that template modules have been adapted from
\cite{GhezziEtAl2003}.  The mathematical notation comes from Chapter 3 of
\citet{HoffmanAndStrooper1995}.  For instance, the symbol := is used for a
multiple assignment statement and conditional rules follow the form $(c_1
\Rightarrow r_1 | c_2 \Rightarrow r_2 | ... | c_n \Rightarrow r_n )$.

The following table summarizes the primitive data types used by \progname. 

\begin{center}
\renewcommand{\arraystretch}{1.2}
\noindent 
\begin{tabular}{l l p{7.5cm}} 
\toprule 
\textbf{Data Type} & \textbf{Notation} & \textbf{Description}\\ 
\midrule
character & char & a single symbol or digit\\
integer & $\mathbb{Z}$ & a number without a fractional component in (-$\infty$, $\infty$) \\
natural number & $\mathbb{N}$ & a number without a fractional component in [1, $\infty$) \\
real & $\mathbb{R}$ & any number in (-$\infty$, $\infty$)\\
policy & Policy & A matrix of size N-by-N, where N is the game state size \\
string & String & A series of characters.  \\
game state & GameState & A matrix of size N-by-N, where N is the game state size \\
simulation & Simulation & A JSON Object that has another JSON object with the following keys: player, turn_num, action, action_details, meta_data. \\
actions & Actions & A JSON Object that has a string as the key and null as the value. The string is the action name. \\
\bottomrule
\end{tabular} 
\end{center}

\noindent
The specification of \progname \ uses some derived data types: sequences, strings, and
tuples. Sequences are lists filled with elements of the same data type. Strings
are sequences of characters. Tuples contain a list of values, potentially of
different types. In addition, \progname \ uses functions, which
are defined by the data types of their inputs and outputs. Local functions are
described by giving their type signature followed by their specification.

\section{Module Decomposition}

The following table is taken directly from the Module Guide document for this project.

\begin{table}[h!]
\centering
\begin{tabular}{p{0.1\textwidth} p{0.2\textwidth} p{0.6\textwidth}}
\toprule
\textbf{Module Type} & \textbf{Module Name} & \textbf{Module Description}\\
\midrule
AI & \multirow{1}{0.3\textwidth}{AI Agent Module} & Trains AI Agents on the game and generates a policy\\
AI & \multirow{1}{0.3\textwidth}{Game Environment Module} & Receives input from AI Agents to take action on the game \\
\midrule
GE & \multirow{1}{0.24\textwidth}{Actions (Commands) Module} & {\textcolor{white}{\_\_\_}}Describes the possible game moves that the AI Agents are able to take \\
GE & \multirow{1}{0.3\textwidth}{Game Loop \\Module} & Continues the game loop for the game and checks if the game-over condition has been fulfilled or not. \\
\midrule

DV & \multirow{1}{0.3\textwidth}{JSON Module} & Responsible for recording each AI Agents moves and observation space and putting them into a JSON file. \\
DV & \multirow{1}{0.3\textwidth}{Graph Module} & Produce a graph selected by the user.  \\
DV & \multirow{1}{0.3\textwidth}{JSON Data Parser \\ Module} & Parses JSON files with AI Agents' move history \\\\
\bottomrule

\end{tabular}
\caption{Module Hierarchy}
\label{TblMH}
\end{table}

\newpage

\section{MIS of AI Agent Module} \label{AIModule} 
\subsection{Module}
AIAgent
\subsection{Uses}
Game Environment Module \ref{GEModule}
\subsection{Syntax}
\subsubsection{Exported Types}
Policy = ?
\subsubsection{Exported Access Programs}

\begin{tabular}{p{2cm} p{4cm} p{4cm} p{5cm}}
\hline
\textbf{Name} & \textbf{In} & \textbf{Out} & \textbf{Exceptions} \\
\hline
run & $\mathbb{N}$ $\mathbb{N}$ $\mathbb{N}$ $\mathbb{N}$ String & Policy & PathDoesNotExist \\
\hline
\end{tabular}

\subsection{Semantics}

\subsubsection{State Variables}
GameEnvironment: GameEnvironment

\subsubsection{Environment Variables}
\noindent Device: The CPU or GPU the system will use to train the AI.

\subsubsection{Assumptions}
The constructor of GameEnvironment is called before the access routine is called.

\subsubsection{Access Routine Semantics}

\noindent run(training-num, test-num, n-step, epoch, resume-path):
\begin{itemize}
\item output: out:= Policy
\item exception: $exc: = \neg$ resume-path $\Rightarrow$ PathDoesNotExist
\end{itemize}

\newpage
\section{MIS of Game Environment Module} \label{GEModule} 
\subsection{Module}
GameEnvironment

\subsection{Uses}
Game Loop \ref{GLModule}

\subsection{Syntax}

\subsubsection{Exported Types}
GameEnvironment = ?

\subsubsection{Exported Access Programs}

\begin{tabular}{p{4cm} p{2cm} p{4cm} p{5cm}}
\hline
\textbf{Name} & \textbf{In} & \textbf{Out} & \textbf{Exceptions} \\
\hline
GameEnvironment &  & GameLog &  \\
observe & $\mathbb{N}$ & Sequence of $\mathbb{N}$, GameState & AgentOutOfBounds \\
legalMoves & $\mathbb{N}$ & Sequence of $\mathbb{N}$ & AgentOutOfBounds\\
step & $\mathbb{N}$ & $\mathbb{N}$ & ActionOutOfBounds, AgentOutOfBounds\\
\hline
\end{tabular}

\subsection{Semantics}

\subsubsection{State Variables}
player: Player \\
engine: Game \\
Rewards: Sequence of $\mathbb{R}$ \\
CurrentAgent: $\mathbb{N}$\\

\subsubsection{Environment Variables}
None

\subsubsection{Assumptions}


\subsubsection{Access Routine Semantics}

\noindent GameEnvironment():
\begin{itemize}
\item transition:= \\Agents:=GameLoop.GetAgents() \\ Rewards:= Sequence of 0 of len(Agents)
\item output: out:= GameLog
\item exception: exc:= len(Agents) $\neg$ =len(Rewards) $\Rightarrow$ RewardOutOfBounds
\end{itemize}

\noindent observe(agent):
\begin{itemize}
\item output: out:= GameLoop.getActionMask(agent), GameLoop.getGameState(agent)
\item exception: exc:= agent $>$ len(Agents) $\Rightarrow$ AgentOutOfBounds
\end{itemize}

\noindent legalMoves(agent):
\begin{itemize}
\item output: out:= GameLoop.getActionMask(agent)
\item exception: exc:= agent $>$ len(Agents) $\Rightarrow$ AgentOutOfBounds
\end{itemize}

\noindent step(action):
\begin{itemize}
\item transition:= \\
GameLoop.playTurn(action, CurrentAgent) \\
CurrentAgent := NextAgent() \\
Rewards[CurrentAgent] := GameLoop.getReward(CurrentAgent)
\item output: out:= GameLoop.isGameOver()
\item exception: exc:= \\
CurrentAgent $>$ len(Agents) $\Rightarrow$ AgentOutOfBounds \\
action $>$ len(GameLoop.getActionMask(agent)) $\Rightarrow$ ActionOutOfBounds \\
\end{itemize}

\subsubsection{Local Functions}
\noindent NextAgent():
\begin{itemize}
\item output: out:= (CurrentAgent + 1) mod len(Agents)
\end{itemize}

\newpage
\section{MIS of Actions (Commands) Module} \label{ActionModule} 
\subsection{Module}
Command

\subsection{Uses}
Game Loop \ref{GLModule}

\subsection{Syntax}

\subsubsection{Exported Types}
Command = ?

\subsubsection{Exported Access Programs}

\begin{tabular}{p{2cm} p{4cm} p{4cm} p{5cm}}
\hline
\textbf{Name} & \textbf{In} & \textbf{Out} & \textbf{Exceptions} \\
\hline
Command & Player, Game Engine & Action &  \\
execute & & & \\
check & &Boolean & \\
\hline
\end{tabular}

\subsection{Semantics}

\subsubsection{State Variables}
player: Player \\
engine: GameLoop \\
action: String \\
action\_details: String\\

\subsubsection{Environment Variables}
None

\subsubsection{Assumptions}
There can be more state variables in the child classes based on the complexity of the action. However, this MIS describes the parent Command module.

\subsubsection{Access Routine Semantics}

\noindent Command(player, engine):
\begin{itemize}
\item transition :=\\
player := player\\
engine := engine
\item output: out:= Action
\item exception: N/A
\end{itemize}

\noindent execute():
\begin{itemize}
\item transition: := engine := engine\_new\_state\\
The transition will be a new state to engine based on the action\\
so only defined semi-formally here as general as possible
\item output: N/A
\item exception: N/A
\end{itemize}

\noindent check():
\begin{itemize}
\item output: out:= Boolean
\item exception: N/A
\end{itemize}

\newpage
\section{MIS of Game Loop Module} \label{GLModule} 
\subsection{Module}
GameLoop

\subsection{Uses}
N/A

\subsection{Syntax}
\subsubsection{Exported Types}
GameLoop = ?

\subsubsection{Exported Access Programs}

\begin{tabular}{p{3cm} p{4cm} p{4cm} p{5cm}}
\hline
\textbf{Name} & \textbf{In} & \textbf{Out} & \textbf{Exceptions} \\
\hline
GameLoop & & &  \\
getAgents & & Sequence of Player Objects & \\
getActionMask & $\mathbb{N}$ & & NoCommandModuleFound, AgentOutOfBounds\\
getGameState & $\mathbb{N}$ & & AgentOutOfBounds\\
getReward & $\mathbb{N}$ & & AgentOutOfBounds\\
playTurn & Command Object, $\mathbb{N}$ & & AgentOutOfBounds\\
checkGameOver & & Boolean & \\
\hline
\end{tabular}

\subsection{Semantics}

\subsubsection{State Variables}
turn\_counter: int \\
% map: Map  \textcolor{red}{if we define the type as Map, do we have to do MIS for Map module as well}\\
agents: Sequence of Players \\
state: Sequence of Sequences of int\\
turn\_state: Enum\\

\subsubsection{Environment Variables}
None

\subsubsection{Assumptions}
There will be more state variables and access routines defined to implement the game rules but they are not listed here since they are game dependent.

\subsubsection{Access Routine Semantics}

\noindent GameLoop(agent\_list, initial\_state):
\begin{itemize}
\item transition :=\\
turn\_counter := 0\\
agents := agent\_list\\
state := initial\_state
\item output: out := GameLoop
\item exception: N/A
\end{itemize}

\noindent getAgents():
\begin{itemize}
\item output: out := agents
\item exception: N/A
\end{itemize}

\noindent getActionMask(agent):
\begin{itemize}
\item output: out := Sequence of Command Objects
\item exception: \\
exc1 := agent $>$ len(Agents) $\Rightarrow$ AgentOutOfBounds \\
exc2 := NoCommandModuleFound
\end{itemize}

\noindent getGameState(agent):
\begin{itemize}
\item output: out := state
\item exception: exc := agent $>$ len(Agents) $\Rightarrow$ AgentOutOfBounds
\end{itemize}

\noindent getReward(agent):
\begin{itemize}
\item output: out := agent.get\_rewards()
\item exception: exc := agent $>$ len(Agents) $\Rightarrow$ AgentOutOfBounds
\end{itemize}

\noindent playTurn(action, agent):
\begin{itemize}
\item transition engine := action.execute()
\item exception: exc := agent $>$ len(Agents) $\Rightarrow$ AgentOutOfBounds
\end{itemize}

\noindent checkGameOver():
\begin{itemize}
\item output out := if state.game\_complete $\Rightarrow$ True else $\Rightarrow$ False
\item exception: exc := agent $>$ len(Agents) $\Rightarrow$ AgentOutOfBounds
\end{itemize}

\newpage
\section{MIS of JSON Module} \label{JSONModule} 
\subsection{Module}
JSON
\subsection{Uses}
Game Environment Module \ref{GEModule}
\subsection{Syntax}
\subsubsection{Exported Types}
JSON = ?
\subsubsection{Exported Access Programs}

\begin{tabular}{p{4cm} p{3cm} p{4cm} p{4cm}}
\hline
\textbf{Name} & \textbf{In} & \textbf{Out} & \textbf{Exceptions} \\
\hline
jsonNamer & String & String &  &
jsonDirectory & String & String &   &
jsonWriter & String, String & JSON & PathDoesNotExist  &
jsonDump & String, String &  & FileDoesNotExist  &
jsonActionConverter & String, Sequence of String & JSON & &
\\
\hline
\end{tabular}

\subsection{Semantics}

\subsubsection{State Variables}
cur\_directory : String \\
cur\_time : String

\subsubsection{Environment Variables}
N/A

\subsubsection{Assumptions}
N/A

\subsubsection{Access Routine Semantics}

\noindent jsonNamer(folder\_name):
\begin{itemize}
\item transition: N/A
\item output: out:= String
\item exception: N/A
\end{itemize}

\noindent jsonDirectory(folder\_path, json\_name):
\begin{itemize}
\item transition: N/A
\item output: out:= String
\item exception:   
\end{itemize}

\noindent jsonWriter(folder\_name, json\_name):
\begin{itemize}
\item transition: N/A
\item output: out:= JSON
\item exception: exc := if path.exists(folder\_path) $\Rightarrow$ False $\Rightarrow$ PathDoesNotExist
\end{itemize}

\noindent jsonDump(simulation\_history, json\_name):
\begin{itemize}
\item transition:= := JSON
\item output: out:= N/A
\item exception: exc := if file.exists(json\_name) $\Rightarrow$ False $\Rightarrow$ FileDoesNotExist
\end{itemize}

\noindent jsonActionConverter(folder\_name, action\_list):
\begin{itemize}
\item transition:= N/A
\item output: out:= JSON
\item exception: exc := if path.exists(folder\_name) $\Rightarrow$ False $\Rightarrow$ PathDoesNotExist
\end{itemize}
\newpage
\section{MIS of Graph Module} \label{GraphModule} 
\subsection{Module}
Graph

\subsection{Uses}
JSONDataParser \ref{DataParserModule}

\subsection{Syntax}
N/A
\subsubsection{Exported Types}
N/A
\subsubsection{Exported Access Programs}

\begin{tabular}{p{2cm} p{4cm} p{4cm} p{5cm}}
\hline
\textbf{Name} & \textbf{In} & \textbf{Out} & \textbf{Exceptions} \\
\hline
render & & HTML &  \\
\hline
\end{tabular}

\subsection{Semantics}

\subsubsection{State Variables}
N/A
\subsubsection{Environment Variables}
N/A

\subsubsection{Assumptions}
This module will render a graph using the JSONDataParser methods. It will be upto the developer to create their custom graphs.

\subsubsection{Access Routine Semantics}

\noindent render():
\begin{itemize}
\item transition := N/A
\item output: out:= HTML
\item exception: N/A
\end{itemize}

\newpage
\section{MIS of JSON Data Parser Module} \label{DataParserModule} 
\subsection{Module}
JsonDataParser

\subsection{Uses}
JSON Module \ref{JSONModule}

\subsection{Syntax}

\subsubsection{Exported Types}
DataParser

\subsubsection{Exported Access Programs}

\begin{tabular}{p{5cm} p{4cm} p{4cm} p{3cm}}
\hline
\textbf{Name} & \textbf{In} & \textbf{Out} & \textbf{Exceptions} \\
\hline
DataParser & Simulation[] & DataParser &  \\
setAllData & Simulation[] & & \\
setAllActions & Actions JSON Object & & \\
getAllData & & Simulation[] & \\
getAllActions & & Actions JSON Object & \\
getAllDataExEnd & Simulation[] & Simulation[]  & \\
getDataWithMergedActions & Simulation[] & Simulation[]  & \\
getSimulationData & Simulation[], $\mathbb{Z}, \mathbb{Z}$ & Simulation[]  & \\
getPlayerData & Simulation, $\mathbb{Z}$ & JSON Object  & \\
getScores & Simulation[], $\mathbb{Z}$, $\mathbb{Z}$  & JSON Object[]  & \\
getNumberOfPlayers & Simulation[] & String[]  & \\
getNumberOfSimulations & Simulation[] & $\mathbb{Z}$  & \\
\hline
\end{tabular}

\subsection{Semantics}

\subsubsection{State Variables}
data: Simulation[]
allActions: Actions JSON Object

\subsubsection{Environment Variables}
None

\subsubsection{Assumptions}
This module assumes that the developer will be able to parse the JSON files properly.

\subsubsection{Access Routine Semantics}

\noindent DataParser()(data):
\begin{itemize}
\item transition := data := data
\item output: out:= self
\item exception: N/A
\end{itemize}

\noindent setAllData(data):
\begin{itemize}
\item transition := data := data
\item output: N/A
\item exception: N/A
\end{itemize}

\noindent setAllActions(allActions):
\begin{itemize}
\item transition := allActions := allActions
\item output: N/A
\item exception: N/A
\end{itemize}

\noindent getAllData():
\begin{itemize}
\item transition := N/A
\item output: data
\item exception: N/A
\end{itemize}

\noindent getAllActions():
\begin{itemize}
\item transition := N/A
\item output: allActions
\item exception: N/A
\end{itemize}

\noindent getAllDataExEnd(inputData):
\begin{itemize}
\item transition := \\
result := []\\
$\forall s: inputData. s < |inputData|-1. sim = \{\}; (\forall i: simulationData[s]. i < |simulationData[s]|-1. simulationData[s][i].action \neq$ "End Turn" $\Rightarrow sim.add(simulationData[s][i])) \Rightarrow result.add(sim)$
\item output: result
\item exception: N/A
\end{itemize}

\noindent getDataWithMergedActions(inputData):
\begin{itemize}
\item transition := \\
result := []\\
$\forall s: inputData. s < |inputData|-1. sim = \{\}; (\forall i: simulationData[s]. i < |simulationData[s]|-1. simulationData[s][i].action \neq$ "meta\_data" $\Rightarrow simulationData[s][i].action = simulationData[s][i].action + simulationData[s][i].action\_detail; sim.add(simulationData[s][i])) \Rightarrow result.add(sim)$
\item output: result
\item exception: N/A
\end{itemize}

\noindent getSimulationData(inputData, startIndex, endIndex):
\begin{itemize}
\item transition := \\
result := []\\
$\forall s: inputData. s >= startIndex \& s < endIndex. \Rightarrow 
result.add(inputData[s])$
\item output: result
\item exception: N/A
\end{itemize}

\noindent getPlayerData(inputData, player):
\begin{itemize}
\item transition := \\
result := \{\}\\
$\forall s: inputData. s < |inputData| - 1 . inputData[s].player = player \Rightarrow 
result.add(s)$
\item output: result
\item exception: N/A
\end{itemize}

\noindent getScores(inputData, startSim, endSim):
\begin{itemize}
\item transition := \\
result := []\\
simulationData := getSimulationData(inputData, startSim, endSim)\\
$\forall s: simulationData. s < |simulationData|-1. sim = \{\}; (\forall i: simulationData[s]. i < |simulationData[s]|-1. i.action = $ "meta\_data" $\Rightarrow sim.add(simulationData[s][i])) \Rightarrow result.add(sim)$
\item output: result
\item exception: N/A
\end{itemize}

\noindent getNumberOfPlayers(inputData):
\begin{itemize}
\item transition := \\
result := []\\
$\exists s: inputData. (\exisits i: inputData[s]. i.action = $ "meta\_data" $\Rightarrow i) \Rightarrow \forall p: i. result.add(p.split($"\_"$)[1])$
\item output: result
\item exception: N/A
\end{itemize}

\noindent getNumberOfSimulations(inputData):
\begin{itemize}
\item transition := \\
result := []\\
$\forall s: inputData. s < |inputData|-1.  \Rightarrow result.add(s)$
\item output: result
\item exception: N/A
\end{itemize}

\section{Appendix} \label{Appendix}

\bibliographystyle {plainnat}
\bibliography {References}


\end{document}