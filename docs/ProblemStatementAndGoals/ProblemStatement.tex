\documentclass{article}

\usepackage{tabularx}
\usepackage{booktabs}

\title{Problem Statement and Goals\\\progname}

\author{\authname}

\date{}

\input{../Comments}
%% Common Parts

\newcommand{\progname}{ProgName} % PUT YOUR PROGRAM NAME HERE %Every program
                                % should have a name

\usepackage{hyperref}
    \hypersetup{colorlinks=true, linkcolor=blue, citecolor=blue, filecolor=blue,
                urlcolor=blue, unicode=false}
    \urlstyle{same}
                                


\begin{document}

\maketitle

\begin{table}[hp]
\caption{Revision History} \label{TblRevisionHistory}
\begin{tabularx}{\textwidth}{llX}
\toprule
\textbf{Date} & \textbf{Developer(s)} & \textbf{Change}\\
\midrule
Date1 & Name(s) & Description of changes\\
Date2 & Name(s) & Description of changes\\
... & ... & ...\\
\bottomrule
\end{tabularx}
\end{table}

\section{Problem Statement}

\wss{You should check your problem statement with the
\href{https://github.com/smiths/capTemplate/blob/main/docs/Checklists/ProbState-Checklist.pdf}
{problem statement checklist}.}
\wss{You can change the section headings, as long as you include the required information.}

\subsection{Problem}

\subsection{Inputs and Outputs}

\wss{Characterize the problem in terms of ``high level'' inputs and outputs.  
Use abstraction so that you can avoid details.}

\subsection{Stakeholders}

\subsection{Environment}

\wss{Hardware and software}

\section{Goals}
\subsection{Automation}
\begin{itemize}
\item Explanation: The product shall automate artificial players to play thousand of games and efficiently detect the loopholes in the game.

\item Importance: This goal ensures high-quality results as each task is performed identically without human error. 
\end{itemize}

\subsection{Customization}
\begin{itemize}
\item Explanation: The product should provide sufficient options for the users to design their level of input by preference and achieve the business needs. 
\item Importance: This goal attempt to provide more control for users and improve the user experience. 
\end{itemize}

\subsection{Interactive UI}
\begin{itemize}
\item Explanation: The user interface of the product should have sufficient elements that users can easy to access, understand and perform the expected tasks from the users. Besides, the UI shall be responsive that dynamically fit different sizes of screens. 
\item Importance: This goal provides good impressions for users and encourages them to use our software product continuously. 
\end{itemize}

\subsection{Short Time Execution}
\begin{itemize}
\item Explanation: The execution time of the product should not be longer than the reasonable maximum value. If the input game state value is invalid, it will terminate the program shortly and report an invalid input error.
\item Importance: This goal tends to improve time efficiency. 
\end{itemize}

\subsection{Analytic and Solving Complex problem}
\begin{itemize}
\item Explanation: The product shall effectively analyze diverse complex problems of the game and provide a detailed analytic output. 
\item Importance: This goal improves the software quality and solves difficult tasks for customers.
\end{itemize}


\section{Stretch Goals}
\subsection{Ease of use}
\begin{itemize}
\item Explanation: The installation process of this product should remain in a minimal time. After installing the program, the end users should understand the manipulation and use the products functionality easily. 
\item Importance: This goal helps users save time and reduce the number of requests for technical support.
\end{itemize}

\subsection{Portability}
\begin{itemize}
\item Explanation: The product can be installed and removed from different operating systems and it shall not damage the game and users' devices.
\item Importance: This goal helps users use the software across different environments. 
\end{itemize}

\subsection{Reusability}
\begin{itemize}
\item Explanation: The users should be able to reuse this software product on multiple games.
\item Importance: This goal improves the quality of the software which reduces additional cost and time on redeveloping a similar software for different games.
\end{itemize}

\subsection{Security}
\begin{itemize}
\item Explanation: The product must protect the privacy of the game data and users' confidential information. 
\item Importance: This goal reduces the risk of data breaches and attacks in the system which tends to prevent unauthorized access to confidential information. 
\end{itemize}


\end{document}