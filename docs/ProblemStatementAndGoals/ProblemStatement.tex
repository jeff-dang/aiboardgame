\documentclass{article}

\usepackage{tabularx}
\usepackage{booktabs}

\title{Problem Statement and Goals\\\progname}

\author{\authname}

\date{}

\input{../Comments}
%% Common Parts

\newcommand{\progname}{ProgName} % PUT YOUR PROGRAM NAME HERE %Every program
                                % should have a name

\usepackage{hyperref}
    \hypersetup{colorlinks=true, linkcolor=blue, citecolor=blue, filecolor=blue,
                urlcolor=blue, unicode=false}
    \urlstyle{same}
                                


\begin{document}

\maketitle

\begin{table}[hp]
\caption{Revision History} \label{TblRevisionHistory}
\begin{tabularx}{\textwidth}{llX}
\toprule
\textbf{Date} & \textbf{Developer(s)} & \textbf{Change}\\
\midrule
Date1 & Name(s) & Description of changes\\
Date2 & Name(s) & Description of changes\\
... & ... & ...\\
\bottomrule
\end{tabularx}
\end{table}

\section{Problem Statement}

\wss{You should check your problem statement with the
\href{https://github.com/smiths/capTemplate/blob/main/docs/Checklists/ProbState-Checklist.pdf}
{problem statement checklist}.}
\wss{You can change the section headings, as long as you include the required information.}

\subsection{Problem}

\subsection{Inputs and Outputs}

\wss{Characterize the problem in terms of ``high level'' inputs and outputs.  
Use abstraction so that you can avoid details.}

\subsection{Stakeholders}

\subsection{Environment}

\wss{Hardware and software}

\section{Goals}
\subsection{Simulation and Learning}
\begin{itemize}
\item Explanation: Implement artificial intelligent players into the system to play thousands of games efficiently to detect patterns and winning strategies.
\item Importance: This is the main focus of the project to test and simulate a board game for quality assurance and balancing before release.
\end{itemize}

\subsection{Efficiency}
\begin{itemize}
\item Explanation: Keep the run time of the simulation and AI game players below a threshold where the system can be easily run on the developer's computers.
\item Importance: This goal will allow all developers to run the simulation on their machines, to get the best data from the system and not have to rent out computing space which would increase costs.
\end{itemize}

\subsection{Complex Problem Solving}
\begin{itemize}
\item Explanation: The system must use AI game players to analyze complex decision paths. Each different AI should have a different decision-making process.
\item Importance: This goal will make sure the output of the system is of quality to the game designers since games will be played intelligently and with different strategies.
\end{itemize}

\subsection*{Quality Data}
\begin{itemize}
  \item Explanation: The system must track and save decisions, game states and patterns throughout the simulations, as well as compare simulations against each other to find winning strategies. The output should be easily readable for non-technical users to understand.
  \item Importance: This goal will make sure the game designers can use the system to balance their game.
\end{itemize}


\section{Stretch Goals}
\subsection{Reusability}
\begin{itemize}
\item Explanation: The system should be highly modular, where different AI Game agents can be swapped in and out to provide different outcomes. The game engineer should also be able
to be swapped in and out. The architecture should allow for a simple game like tic-tac-toe and a complex game like An Age Contrived to fit seamlessly into the system.
\item Importance: This goal improves the quality of the code and should be a side-effect of the architecture. This would make the system more useful as it can test more board games than it was initially designed for.
\end{itemize}

\subsection{User Interface}
\begin{itemize}
\item Explanation: The system should have a live user interface where a user can view a game in real-time, to see what actions the AI game players are making.
\item Importance: This goal will make the system easier to use and allow non-technical users to verify it is functional.
\end{itemize}

\subsection{Availability}
\begin{itemize}
  \item Explanation: The system should be deployed on a server, where the it can run for a longer period of time to gather data.
  \item Importance: This goal will allow us to collect more data which can help further analysis of game balancing.
\end{itemize}
\end{document}
